\documentclass[12pt,a4paper,uplatex]{jsarticle}

\usepackage[utf8]{inputenc}
\usepackage{graphicx}
\usepackage[usenames]{xcolor}
\usepackage[varg]{txfonts}
\usepackage{newtxtext}
\usepackage{ulem}
\usepackage{ifthen}
\definecolor{pdarkblue}{rgb}{0.1797, 0.1875, 0.5703}
\usepackage[dvipdfmx]{hyperref}
\hypersetup{
    colorlinks=true,
    citecolor=blue,
    linkcolor=blue,
    urlcolor=blue,
}
\usepackage{pxjahyper} % しおりの文字化けを防ぐ


%========== Templates ==========
% \newcommand{\arXiv}[4]{#1\\``#2''\\\frenchspacing\href{https://arxiv.org/abs/#3}{arXiv:#3} (#4).\nonfrenchspacing}
\newcommand{\arXiv}[4]{\textbf{``#2''}\\#1\\\frenchspacing\href{https://arxiv.org/abs/#3}{arXiv:#3} (#4).\nonfrenchspacing}
% Usage: \arXiv{Authors}{Title}{Identifier}{Year}

% \newcommand{\publication}[6]{#1\\``#2''\\\frenchspacing\href{https://doi.org/#4}{#3} (#5). [\href{https://arxiv.org/abs/#6}{arXiv:#6}]\nonfrenchspacing}
\newcommand{\publication}[6]{\textbf{``#2''}\\#1\\\frenchspacing\href{https://doi.org/#4}{#3} (#5). [\href{https://arxiv.org/abs/#6}{arXiv:#6}]\nonfrenchspacing}
% Usage: \publishd{Authors}{Title}{Bibliography}{doi}{Year}{ArXivIdentifier}



% \newcommand{\ConferenceJP}[7]{#1\\「#2」(#7)\\#3、#4、#6}
% \newcommand{\ConferenceEN}[7]{#1\\``#2'' (#7)\\#3, #4, #6.}
\newcommand{\ConferenceJP}[7]{「#2」(#7)\\#3、#4、#6}
\newcommand{\ConferenceEN}[7]{``#2'' (#7)\\#3, #4, #6.}

\newcommand{\ifempty}[2]{\if\relax\detokenize{#1}\relax\else#1#2\fi}

\newcommand{\SeminarJP}[7]{「#2」\\\ifempty{#3}{、}#4、#6}
\newcommand{\SeminarEN}[7]{``#2''\\\ifempty{#3}{, }#4, #6.}

% Usage:
% \ConferenceJP{Author}{Title}{Conference}{Period}{Date}{Venue}{Type}
% \ConferenceEN{Author}{Title}{Conference}{Period}{Date}{Venue}{Type}



%========== Formatting ==========
\usepackage[top=2.5cm, bottom=2.5cm, left=2.5cm, right=2.5cm]{geometry}
\setlength{\parindent}{0pt}
\renewcommand{\baselinestretch}{0.8}  
\setlength{\parskip}{5pt}
\setlength{\leftskip}{8pt}

\setcounter{secnumdepth}{0}
\usepackage{titlesec}
\titleformat{\section}[hang]{\large\scshape\textgt}{}{0pt}{}[\vspace{-2pt}{\titlerule[0.6pt]}]
\titlespacing{\section}{0pt}{8pt}{2pt}
\titleformat{\subsection}[hang]{\scshape\bfseries\textgt}{}{0pt}{}[]
\titlespacing{\subsection}{0pt}{4pt}{-4pt}



%========== Multi Language ==========
\newcounter{mylanguage}
% \setcounter{mylanguage}{0} %English
\setcounter{mylanguage}{1} %Japanese


\ifodd\value{mylanguage}
\newcommand{\je}[2]{#1}
\else
\newcommand{\je}[2]{#2}
\usepackage[english]{datetime2}
\fi 

%========== Main ==========
\begin{document}
\begin{center}
\Huge \textbf{\je{大賀 成朗}{Naruo Ohga}}

\LARGE {\je{Naruo Ohga}{\vspace{0.3\baselineskip}}}

\normalsize \je{(最終更新:\today)}{(Last modified: \today)}
\end{center}

\vspace{\baselineskip}

{
\setlength{\leftskip}{0pt}
\je{東京大学理学系研究科物理学専攻 博士課程3年。非平衡熱力学の理論を研究しています。}
{Third-year Ph.D.~student at Department of Physics, Graduate School of Science, the University of Tokyo. I study the theoretical aspects of nonequilibrium thermodynamics.}
\par
}


\section{\je{連絡先 / リンク}{Contact \& Links}}

E-mail:
\href{mailto:naruo.ohga@ubi.s.u-tokyo.ac.jp}{naruo.ohga@ubi.s.u-tokyo.ac.jp} \\
% GitHub:
% \href{https://github.com/naruo-ohga}{@naruo-ohga}\\
ORCID iD:
\href{https://orcid.org/0000-0002-1596-6284}{0000-0002-1596-6284}\\
Google Scholar:
\je{\href{https://scholar.google.com/citations?user=jOdc58QAAAAJ&hl=ja}{Link}}{\href{https://scholar.google.com/citations?user=jOdc58QAAAAJ}{Link}} \\
Research map:
\href{https://researchmap.jp/naruo_ohga}{naruo\_ohga}\\
Webpage:
\href{https://naruo-ohga.github.io}{https://naruo-ohga.github.io}


\section{\je{キーワード}{Keywords}}
\je{非平衡熱力学、ゆらぎの熱力学、確率過程、情報理論}{Nonequilibrium thermodynamics, Stochastic thermodynamics, Stochastic processes, Information theory}





\section{\je{経歴}{Education}}


\je{\textbf{東京大学大学院} 博士課程 2023年4月〜現在}{\textbf{The University of Tokyo} (Tokyo, Japan), Apr.~2023--present}\\
\je{理学系研究科 物理学専攻}{Doctoral Program, Department of Physics, Graduate School of Science}

\je{\textbf{東京大学大学院} 修士課程 2021年4月〜2023年3月}{\textbf{The University of Tokyo} (Tokyo, Japan), Apr.~2021--Mar.~2023}\\
\je{理学系研究科 物理学専攻}{Master's Program, Department of Physics, Graduate School of Science}\\
\je{学位:\textbf{修士(理学)} 2023年3月23日}{Degree Title: \textbf{Master of Science} (Mar.~23, 2023)}\\
\je{学位論文:「非平衡熱力学における情報幾何学の大域的構造とその応用」}{Thesis: ``Global structure in information geometry for nonequilibrium thermodynamics and its applications''}

\je{\textbf{東京大学} 2017年4月〜2021年3月}{\textbf{The University of Tokyo} (Tokyo, Japan), Apr.~2017--Mar.~2021}\\
\je{理学部 物理学科}{Bachelor's Program, Physics, Faculty of Science}\\
\je{学位:\textbf{学士(理学)} 2021年3月18日}{Degree Title: \textbf{Bachelor of Science} (Mar.~18, 2021)}\\
\je{[理学部学修奨励賞]}{[The School of Science Encouragement Award for the Undergraduate Program]}


\section{\je{論文}{Research Articles}}


\subsection{\je{出版論文}{Publications}}


\publication
{\emph{Naruo Ohga} and Sosuke Ito}
{Inferring nonequilibrium thermodynamics from tilted equilibrium using information-geometric Legendre transform}
{Phys. Rev. Research \textbf{6}, 013315}
{10.1103/PhysRevResearch.6.013315}
{2024}
{2112.11008}

\publication
{Artemy Kolchinsky, \emph{Naruo Ohga}, and Sosuke Ito}
{Thermodynamic bound on spectral perturbations, with applications to oscillations and relaxation dynamics}
{Phys. Rev. Research \textbf{6}, 013082}
{10.1103/PhysRevResearch.6.013082}
{2024}
{2304.01714}

\publication
{\emph{Naruo Ohga}, Sosuke Ito, and Artemy Kolchinsky}
{Thermodynamic Bound on the Asymmetry of Cross-Correlations}
{Phys. Rev. Lett. \textbf{131}, 077101}
{10.1103/PhysRevLett.131.077101}
{2023}
{2303.13116}
\\{}[Editors' Suggestion] [Highlighted in \href{https://physics.aps.org/articles/v16/142}{Physics Magazine Viewpoint}] \je{[プレスリリース(\href{https://www.s.u-tokyo.ac.jp/ja/press/2023/8610/}{日本語}/\href{https://www.s.u-tokyo.ac.jp/en/press/2023/8524/}{英語})]}{[Press Release (\href{https://www.s.u-tokyo.ac.jp/en/press/2023/8524/}{EN}/\href{https://www.s.u-tokyo.ac.jp/ja/press/2023/8610/}{JP})]}

\publication
{\emph{Naruo Ohga} and Sosuke Ito}
{Information-geometric structure for chemical thermodynamics: An explicit construction of dual affine coordinates}
{Phys. Rev. E \textbf{106}, 044131}
{10.1103/PhysRevE.106.044131}
{2022}
{2112.13813}




\subsection{\je{プレプリント}{Preprints}}

\arXiv
{Xin-Hai Tong, Kohei Yoshimura, Tan Van Vu, and \emph{Naruo Ohga}}
{Interplay between Standard Quantum Detailed Balance and Thermodynamically Consistent Entropy Production}
{2512.06707}
{2025}

\arXiv
{Ruicheng Bao, \emph{Naruo Ohga}, and Sosuke Ito}
{Measuring irreversibility by counting: a random coarse-graining framework}
{2508.11586}
{2025}

\arXiv
{\emph{Naruo Ohga} and Takuya Hatomura}
{Improving variational counterdiabatic driving with weighted actions and computer algebra}
{2505.18367}
{2025}

\arXiv
{Xin Wang, Ruicheng Bao, and \emph{Naruo Ohga}}
{Characteristic oscillations in frequency-resolved heat dissipation of linear time-delayed Langevin systems: Approach from the violation of the fluctuation--response relation}
{2501.01151}
{2025}

\arXiv
{\emph{Naruo Ohga}, Hisao Hayakawa, and Sosuke Ito}
{Microscopic theory of Mpemba effects and a no-Mpemba theorem for monotone many-body systems}
{2410.06623}
{2024}

% \je{なし}{None}



\section{\je{発表}{Presentations}}

% \je{英文は国際会議、和文は国内会議を表す}{}

\subsection{\je{招待講演}{Invited talks}}


\je{
\SeminarJP
{\emph{大賀成朗}}
{駆動力(cycle affinity)に基づく原理限界の話題}
{沙川情報エネルギー変換プロジェクトと情報物理学でひもとく生命の秩序と設計原理の合同勉強会}
{2024年6月6日〜7日}
{2024年6月7日}
{東京大学、東京}
{Seminar}
}{
\SeminarEN
{\emph{Naruo Ohga}}
{Topics on fundamental bounds based on the driving force (cycle affinity)} %英文タイトル提出なし
{Joint workshop between ``ERATO Sagawa information-to-energy interconversion project'' and ``Information physics of living matters''} %正式名称不明
{Jun.~6--7, 2024}
{Jun.~7, 2024}
{The University of Tokyo, Tokyo}
{Seminar}
}



\subsection{\je{セミナー}{Seminars}}



\je{
\SeminarJP
{\emph{大賀成朗}}
{二時刻相関関数と熱力学コストの普遍的・定量的関係}
{UBIミーティング}
{2023年11月8日}
{2023年11月8日}
{東京大学生物普遍性研究機構(UBI)、東京}
{Seminar}
}{
\SeminarEN
{\emph{Naruo Ohga}}
{Universal and quantitative relations between two-time correlations and thermodynamic costs} %英語タイトル提出せず
{UBI meeting}
{Nov.~8, 2023}
{Nov.~8, 2023}
{Universal Biology Institute, The University of Tokyo, Tokyo, Japan}
{Seminar}
}

\je{
\SeminarJP
{\emph{大賀成朗}}
{二時刻相関関数と非平衡駆動強度を結ぶ普遍不等式}
{}
{2023年10月20日}
{2023年10月20日}
{NTT物性科学基礎研究所、神奈川}
{Seminar}
}{
\SeminarEN
{\emph{Naruo Ohga}}
{Universal inequalities connecting two-time correlations and the strength of nonequilibrium driving} %英語タイトル提出せず
{}
{Oct.~20, 2023}
{Oct.~20, 2023}
{NTT Basic Research Laboratories, Kanagawa, Japan}
{Seminar}
}


\je{
\SeminarJP
{\emph{大賀成朗}}
{Universal thermodynamic bounds on two-time correlations}
{}
{2023年6月20日}
{2023年6月20日}
{京都大学、京都}
{Seminar}
}{
\SeminarEN
{\emph{Naruo Ohga}}
{Universal thermodynamic bounds on two-time correlations}
{}
{Jun.~20, 2023}
{Jun.~20, 2023}
{Kyoto University, Kyoto, Japan}
{Seminar}
}


\SeminarEN
{\emph{Naruo Ohga}}
{Information-geometric duality between nonequilibrium states and tilted equilibrium states}
{Joint-group meeting between U. Washington (Qian group) and UNC-Chapel Hill (Lu group)}
{Jul.~14th, 2022}
{Jul.~14th, 2022}
{University of Washington and UNC-Chapel Hill, US (Online)}
{Seminar}



\subsection{\je{国際会議}{International Conferences}}


\ConferenceEN
{\emph{Naruo Ohga}, Hisao Hayakawa, Sosuke Ito}
{Universal microscopic theory of Mpemba effects and a rigorous no-Mpemba theorem}
{STATPHYS29}
{Jul.~13--18, 2025}
{Jul.~14, 2025}
{Palazzo dei Congressi \& Palaffari, Florence, Italy}
{Poster}

\ConferenceEN
{\emph{Naruo Ohga}}
{Universality in cycle affinity: Driving force limits two-time correlations}
{Leuven school: Basics of nonequilibrium statistical mechanics}
{May~19--23, 2025}
{May~22, 2025}
{Aula Arenbergkasteel, KU Leuven, Leuven, Belgium}
{Oral}

\ConferenceEN
{\emph{Naruo Ohga}}
{Universal microscopic theory of Mpemba effects for general classical systems}
{1st India-Japan Workshop on Physical Aspects of Living Systems}
{Feb.~19--21, 2025}
{Feb.~19--21, 2025}
{Mishima Hall, ELSI, Institute of Science Tokyo, Tokyo, Japan}
{Poster}


\ConferenceEN
{\emph{Naruo Ohga}}
{How hot cools faster than cold: Universal microscopic theory of Mpemba effects}
{FoPM International Symposium}
{Feb.~17--19, 2025}
{Feb.~17}
{Ito International Research Center and Yayoi Auditorium, The University of Tokyo, Tokyo, Japan}
{Oral}


\ConferenceEN
{\emph{Naruo Ohga}, Sosuke Ito}
{Legendre duality between nonstationary and equilibrium entropy and its application to thermodynamic inference}
{STATPHYS28}
{Aug.~7--11, 2023}
{Aug.~7}
{Hongo campus, The University of Tokyo, Tokyo, Japan}
{Oral}


\makeatletter
\ConferenceEN
{\emph{Naruo Ohga}}
{Thermodynamic bound on the asymmetry of cross-correlations}
{YITP-YSF Symposium ``Perspectives on Non-Equilibrium Statistical Mechanics: The 45th Anniversary Symposium of Yamada Science Foundation,''\@gobble}
{Aug.~3--5, 2023}
{Aug.~4}
{Panasonic Auditorium, Yukawa Hall, Yukawa Institute for Theoretical Physics, Kyoto University, Kyoto, Japan}
{Oral}
\makeatother


\ConferenceEN
{\emph{Naruo Ohga}}
{Universal relations on nonequilibrium entropy in classical fluctuating systems}
{FoPM International Symposium}
{Feb.~6--8, 2023}
{Feb.~7, 2023}
{Ito Hall, The University of Tokyo, Tokyo, Japan}
{Poster}


\ConferenceEN
{\emph{Naruo Ohga}}
{Legendre duality in stochastic thermodynamics: A construction based on information geometry}
{Workshop on Stochastic Thermodynamics III}
{May~26--Jun.~3, 2022}
{May~30}
{Japan (Online)}
{Oral}





\subsection{\je{国内学会等}{Conferences in Japan}}



\je{
\ConferenceJP
{\emph{大賀成朗}、鳩村拓矢}
{重みつき変分関数と計算機代数に基づく、局所制御項を用いる断熱ショートカットの改良}
{日本物理学会第80回年次大会}
{2025年9月16日〜19日}
{9月17日}
{広島大学 東広島キャンパス、広島}
{Oral}
}{
\ConferenceEN
{\emph{Naruo Ohga}, Takuya Hatomura}
{Improving variational counterdiabatic driving based on weighted actions and computer algebra}
{80th Annual Meeting, The Physical Society of Japan}
{Sep.~16--19, 2025}
{Sep.~17}
{Hiroshima University (Higashi-Hiroshima Campus), Hiroshima, Japan}
{Oral}
}
%17pSK102-11


\je{
\ConferenceJP
{\emph{大賀成朗}}
{ゆらぐ古典系におけるㇺペンバ効果の普遍的なミクロ理論}
{ERATO・学術変革B合同合宿会議}
{2025年3月26日〜28日}
{3月27日}
{ふくしま 磐梯熱海温泉 ホテル華の湯、福島}
{Poster}
}{
\ConferenceEN
{\emph{Naruo Ohga}}
{Universal microscopic theory of Mpemba effects in fluctuating classical systems} %英文タイトル提出せず
{ERATO \& Gakujutsu-henkaku B Joint Meeting} %正確な英訳は不明
{Mar.~26--28, 2025}
{Mar.~27}
{Fukushima Bandai-atami-onsen Hotel Hananoyu, Fukushima, Japan}
{Poster}
}


\je{
\ConferenceJP
{\emph{大賀成朗}、早川尚男、伊藤創祐}
{ミクロ状態どうしの比較によるMpemba効果の解析の一般論}
{日本物理学会第79回年次大会}
{2024年9月16日〜19日}
{9月19日}
{北海道大学 札幌キャンパス、北海道}
{Oral}
}{
\ConferenceEN
{\emph{Naruo Ohga}, Hisao Hayakawa, Sosuke Ito}
{General framework for analyzing Mpemba effects by comparing microstates}
{79th Annual Meeting, The Physical Society of Japan}
{Sep.~16--19, 2024}
{Sep.~19}
{Hokkaido University (Sapporo Campus), Hokkaido, Japan}
{Oral}
}
%19aE313-4


\makeatletter

\je{
\ConferenceJP
{\emph{大賀成朗}}
{相互相関関数の非対称性に関する熱力学的限界の研究(受賞講演)}
{新学術領域研究「情報物理学でひもとく生命の秩序と設計原理」第8回領域会議}
{2024年3月4日〜5日}
{2024年3月5日}
{東京大学鉄門記念講堂、東京}
{Oral}
}{
\ConferenceEN
{\emph{Naruo Ohga}}
{Research on thermodynamic bounds on the asymmetry of cross-correlations (award talk)} %英語タイトル提出せず
{8th Conference, Grant-in-Aid for Scientific Research on Innovative Areas: ``Information physics of living matters,''\@gobble}
{Mar.~4--5, 2024}
{Mar.~5, 2024}
{Tetsumon Memorial Hall, The University of Tokyo, Tokyo, Japan}
{Oral}
}

\makeatother


\makeatletter

\je{
\ConferenceJP
{\emph{大賀成朗}}
{二時刻相関関数にひそむ熱力学コスト}
{新学術領域研究「情報物理学でひもとく生命の秩序と設計原理」第7回領域会議}
{2023年9月21日〜22日}
{9月21日}
{朱鷺メッセ 新潟コンベンションセンター、新潟}
{Oral}
}{
\ConferenceEN
{\emph{Naruo Ohga}}
{Thermodynamic costs behind two-time correlations}
{7th Conference, Grant-in-Aid for Scientific Research on Innovative Areas: ``Information physics of living matters,''\@gobble}
{Sep.~21--22, 2023}
{Sep.~21}
{Toki Messe Niigata Convention Center, Niigata, Japan}
{Oral}
} % 英語題目の登録なし



\je{
\ConferenceJP
{\emph{大賀成朗}}
{相関関数に対する熱力学的制約と生命の熱力学コストへの適用可能性}
{新学術領域研究「情報物理学でひもとく生命の秩序と設計原理」第7回領域会議}
{2023年9月21日〜22日}
{9月21日}
{朱鷺メッセ 新潟コンベンションセンター、新潟}
{Poster}
}{
\ConferenceEN
{\emph{Naruo Ohga}}
{Thermodynamic bounds on correlation functions and their applicability to biological energetic costs}
{7th Conference, Grant-in-Aid for Scientific Research on Innovative Areas: ``Information physics of living matters,''\@gobble}
{Sep.~21--22, 2023}
{Sep.~21}
{Toki Messe Niigata Convention Center, Niigata, Japan}
{Poster}
} % 英語題目の登録なし


\je{
\ConferenceJP
{\emph{大賀成朗}、伊藤創祐}
{外場下の平衡測定を用いた非定常過程の熱力学推定手法}
{日本物理学会第78回年次大会}
{2023年9月16日〜19日}
{9月17日}
{東北大学 青葉山・川内キャンパス、宮城}
{Oral}
}{
\ConferenceEN
{\emph{Naruo Ohga}, Sosuke Ito}
{Thermodynamic inference of nonstationary processes from tilted equilibrium measurements}
{78th Annual Meeting, The Physical Society of Japan}
{Sep.~16--19, 2023}
{Sep.~17}
{Tohoku University (Aobayama Campus, Kawauchi Campus), Miyagi, Japan}
{Oral}
}
% No.~17aB203-12
% 物性は川内キャンパス



\je{
\ConferenceJP
{\emph{大賀成朗}}
{時間発展生成子の固有値に対する熱力学的限界}
{第68回物性若手夏の学校}
{2023年8月12日〜15日}
{8月13日}
{奥琵琶湖マキノパークホテル\&セミナーハウス、滋賀}
{Oral}
}{
\ConferenceEN
{\emph{Naruo Ohga}}
{Thermodynamic bound on the eigenvalues of the time
evolution generator}
{The 68th Condensed Matter Physics Summer School}
{Aug.~12--15, 2023}
{Aug.~13}
{Makino Parkhotel \& Seminarhouse, Shiga, Japan}
{Oral}
}



\je{
\ConferenceJP
{\emph{大賀成朗}、伊藤創祐、Artemy Kolchinsky}
{相互相関関数に対する熱力学的制約と振動固有値への応用}
{日本物理学会2023年春季大会}
{2023年3月22日〜25日}
{3月24日}
{オンライン}
{Oral}
}{
\ConferenceEN
{\emph{Naruo Ohga}, Sosuke Ito, Artemy Kolchinsky}
{Thermodynamic bound on the cross-correlations and its application to oscillatory eigenvalues}
{2023 Spring Meeting, The Physical Society of Japan}
{Mar.~22--25, 2023}
{Mar.~24}
{Japan (Online)}
{Oral}
}
% No.~24aL2-9



\je{
\ConferenceJP
{\emph{大賀成朗}}
{相関関数の時間反転対称性の破れに対する熱力学的制約}
{新学術領域研究「情報物理学でひもとく生命の秩序と設計原理」第6回領域会議}
{2023年3月6日〜7日}
{3月6日}
{アクロス福岡 国際会議場、福岡}
{Poster}
}{
\ConferenceEN
{\emph{Naruo Ohga}}
{Thermodynamic bound on the time-reversal symmetry breaking in cross-correlations}
{6th Conference, Grant-in-Aid for Scientific Research on Innovative Areas: ``Information physics of living matters,''\@gobble}
{Mar.~6--7, 2023}
{Mar.~6}
{ACROS Fukuoka International Conference Hall, Fukuoka, Japan}
{Poster}
} % 英語題目の登録なし



\je{
\ConferenceJP
{\emph{大賀成朗}}
{ゆらぐ系における非平衡状態と外力下の平衡状態との熱力学的双対関係}
{第67回物性若手夏の学校}
{2022年8月2日〜5日}
{8月3日}
{オンライン}
{Oral}
}{
\ConferenceEN
{\emph{Naruo Ohga}}
{Thermodynamic duality between nonequilibrium states and tilted equilibrium states in stochastic systems}
{The 67th Condensed Matter Physics Summer School}
{Aug.~2--5, 2022}
{Aug.~3}
{Japan (Online)}
{Oral}
}




\je{
\ConferenceJP
{\emph{大賀成朗}}
{ゆらぐ系における非平衡緩和過程と外力下の平衡準静過程との双対関係}
{新学術領域研究「情報物理学でひもとく生命の秩序と設計原理」第5回領域会議}
{2022年6月20日〜21日}
{6月20日}
{淡路夢舞台国際会議場、兵庫}
{Poster}
}{
\ConferenceEN
{\emph{Naruo Ohga}}
{Thermodynamic duality between nonequilibrium relaxation processes and tilted quasi-static processes in stochastic systems}
{5th Conference, Grant-in-Aid for Scientific Research on Innovative Areas: ``Information physics of living matters,''\@gobble}
{Jun.~20--21, 2022}
{Jun.~20}
{Awaji Yumebutai International Conference Center, Hyogo, Japan}
{Poster}
} % 英語題目の登録なし



\je{
\ConferenceJP
{\emph{大賀成朗}、伊藤創祐}
{ゆらぎの熱力学における情報幾何的なルジャンドル双対性}
{日本物理学会第77回年次大会}
{2022年3月15日〜19日}{3月16日}
{オンライン}
{Oral}
}{
\ConferenceEN
{\emph{Naruo Ohga}, Sosuke Ito}
{Legendre duality in stochastic thermodynamics based on information geometry}
{77th Annual Meeting, The Physical Society of Japan}
{Mar.~15--19, 2022}
{Mar.~16}
{Japan (Online)}
{Oral}
}
% No.~16aB14-4

\je
{\ConferenceJP{\emph{大賀成朗}、伊藤創祐}{非平衡化学熱力学の情報幾何学的な双対座標系}{日本物理学会2021年秋季大会}{2021年9月20日〜23日}{9月22日}{オンライン}{Oral}}
{\ConferenceEN{\emph{Naruo Ohga}, Sosuke Ito}{Information-geometric dual affine coordinates in non-equilibrium chemical thermodynamics}{2021 Autumn Meeting, The Physical Society of Japan}{Sep.~20--23, 2021}{Sep.~22}{Japan (Online)}{Oral}}
% No.~22pL1-15

\je
{\ConferenceJP{\emph{大賀成朗}}{双対平坦幾何による確率熱力学の大域的構造の一例}{第66回物性若手夏の学校}{2021年8月2日〜5日}{8月3日}{オンライン}{Oral}}
{\ConferenceEN{\emph{Naruo Ohga}}{A global structure of stochastic thermodynamics based on dually flat geometry}{The 66th Condensed Matter Physics Summer School}{Aug.~2--5, 2021}{Aug.~3}{Japan (Online)}{Oral}}


% \section{\je{外部資金}{Grants}}

% \section{\je{受賞}{Awards}}

% \section{\je{採用}{Fellowships}}


\section{\je{受賞・採用}{Awards \& Fellowships}}

%!!!!!
%受賞にはタイトルを付すのが正式らしい。Research mapを参照。
%!!!!!

\je{2024年3月5日}{Mar.~5, 2024}
---
\je{新学術領域研究「情報物理学でひもとく生命の秩序と設計原理」\href{https://infophys-bio.jp/award}{第4回研究賞}}
{\href{https://infophys-bio.jp/award}{Research award}, Grant-in-Aid for Scientific Research on Innovative Areas: ``Information physics of living matters''}
% "Research award" は受賞記念品の刻印から


\je{2023年9月22日}{Sep.~22, 2023}
---
\je{新学術領域研究「情報物理学でひもとく生命の秩序と設計原理」\href{https://infophys-bio.jp/archives/1035}{第7回領域会議 発表賞}}
{\href{https://infophys-bio.jp/archives/1035}{Presentation award}, 7th Conference, Grant-in-Aid for Scientific Research on Innovative Areas: ``Information physics of living matters''}
% "Presentation award" は試訳。正式名称不明


\je{2023年8月16日}{Aug.~16, 2023}
---
\je{Physical Review Letters誌~ \href{https://journals.aps.org/prl/abstract/10.1103/PhysRevLett.131.077101}{Editors' Suggestion}}{\href{https://journals.aps.org/prl/abstract/10.1103/PhysRevLett.131.077101}{Editors' Suggestion}, Physical Review Letters}


\je{2023年4月〜2026年3月}{Apr.~2023--Mar.~2026} 
--- 
\je{日本学術振興会 \href{https://www.jsps.go.jp/j-pd/index.html}{特別研究員(DC1)}}{\href{https://www.jsps.go.jp/english/e-pd/index.html}{Research Fellowship for Young Scientists (DC1)}, Japan Society for the Promotion of Science (JSPS).}

\je{2021年4月〜2026年3月}{Apr.~2021--Mar.~2026} 
--- 
\je{東京大学国際卓越大学院教育プログラム(WINGS)「\href{https://www.s.u-tokyo.ac.jp/ja/FoPM/}{変革を駆動する先端物理・数学プログラム(FoPM)}」}{\href{https://www.s.u-tokyo.ac.jp/en/FoPM/}{Forefront Physics and Mathematics Program to Drive Transformation (FoPM)}, a World-leading Innovative Graduate Study (WINGS) Program, The University of Tokyo.}


\je{2021年3月18日}{Mar.~18, 2021} 
---
\je{東京大学理学部 \href{https://www.phys.s.u-tokyo.ac.jp/award/27300}{理学部学修奨励賞}}{\href{https://www.phys.s.u-tokyo.ac.jp/award/27300}{The School of Science Encouragement Award for the Undergraduate Program}, School of Science, The University of Tokyo}


% \section{\je{教育}{Teaching}}


% \subsection{\je{ティーチングアシスタント (TA)}{Teaching assistant}}
% \je{2021年秋学期「非平衡科学」}{Autumn 2021, ``Non-equilibrium Physics''}


\section{\je{アウトリーチ}{Outreach \& Popular science}}


\je{2024年5月}{May 2024} --- 
\je{東京大学理学部 \href{https://www.s.u-tokyo.ac.jp/ja/story/newsletter/page/10353}{理学部ニュース56巻1号} 記事「不等式の数学で探る,生命機能の物理的制約」}{Article ``Physical constraints on biological functions, explored by the mathematics of inequalities'' (in Japanese), \href{https://www.s.u-tokyo.ac.jp/ja/story/newsletter/page/10353}{Rigakubu News Vol.~56, No.~1}, School of Science, the University of Tokyo}
% 英文題目登録なしだが、https://www.s.u-tokyo.ac.jp/en/story/newsletter/page/10348/ に機械翻訳が載っている。ここではそれを少し改変したものを掲載


\je{2020年5月}{May 2020} --- 
\je{東京大学五月祭 理学部物理学科 学生展示 \href{https://event.phys.s.u-tokyo.ac.jp/physlab2020/index.html}{Physics Lab 2020} 生物物理班・量子情報班}
{Biophysics Group \& Quantum Information Group in \href{https://event.phys.s.u-tokyo.ac.jp/physlab2020/index.html}{Physics Lab 2020},
Students' presentation at the school festival of the University of Tokyo} 


\je{2019年5月}{May 2019} --- \je{東京大学五月祭 理学部物理学科 学生展示 \href{https://event.phys.s.u-tokyo.ac.jp/physlab2019/}{Physics Lab 2019} 量子情報班}
{Quantum Information Group in \href{https://event.phys.s.u-tokyo.ac.jp/physlab2019/}{Physics Lab 2019},
Students' presentation at the school festival of the University of Tokyo} 



\section{\je{スキル}{Skills}}
\je{プログラミング:}{Programming: }Python (numpy, sympy, matplotlib, pandas), C++ (modern)\\
\je{ソフトウェア / サービス:}{Software \& Service: }\LaTeX, Adobe Illustrator

\je{言語:}{Languages: }\je{日本語(母語)、英語(上級)、フランス語(初級)}{Japanese (native), English (fluent), French (basic)}


\end{document}